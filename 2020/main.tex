%!TEX TS-program = xelatex
\documentclass[a4paper, 10.5pt]{article}
\usepackage{comment} % enables the use of multi-line comments (\ifx \fi) 
\usepackage{fullpage} % changes the margin
\usepackage[margin=35pt]{geometry}
\usepackage{graphicx, float, caption}
\usepackage{microtype}
\usepackage{hyperref}
\usepackage{titlesec}
\usepackage{color, colortbl}
\usepackage[table]{xcolor} 
\usepackage[numbers,super]{natbib}
\usepackage{bibentry}
\nobibliography*

%\usepackage[most]{tcolorbox}

\usepackage{fontspec}
\setmainfont{Palatino} 
\titleformat*{\section}{\large\bfseries}

\begin{document}

\noindent
\large\textbf{Committee Meeting Report} \hfill \textbf{Arjun Biddanda} \\
\normalsize  \hfill PI: John Novembre  \\
HG Matriculated 2015 \hfill Date: 6/16/20 \\
\noindent\makebox[\linewidth]{\rule{\paperwidth}{0.4pt}}

\large\textbf{\\Progress since last Committee Meeting - 11/4/19}

\subsection*{Major deadlines in 2020}
\begin{itemize}
    \item \textbf{Fall Dissertation Submission Deadline: 11/13/20}
    \item \textbf{Target Dissertation Defense Date: 11/5/20}
    \item Postdoc offer starting in early 2021 (at Oxford with Prof. Pier Palamara)
\end{itemize}

\subsection*{Manuscripts in progress}
\begin{itemize}
		\item Preprint + submitted: \bibentry{Biddanda2020a}
    \item Draft: \bibentry{Biddanda2020b}
    \item Paper outline \& initial draft: \bibentry{Biddanda2020c}
\end{itemize}

\section*{Dissertation progress since my last committee meeting:}
 \textbf{AIM 1:} Characterizing the geographic distribution of genetic variants in variant sets \\
 \begin{itemize}
   \item Updated results and submitted manuscript to \texttt{eLife}. (preprint on \texttt{bioRxiv}).
   \item Repository to reproduce figures has been generated and can be found \href{https://github.com/aabiddanda/geovar\_rep\_paper}{here}
 \end{itemize}
 
 \noindent \textbf{AIM 2:} Investigating the effects of serial sampling on two-locus coalescent properties and linkage disequilibrium\\
  \begin{itemize}
    \item Theory developed largely in two contexts: (1) two-locus genealogies and (2) haplotype-copying models with serial sampling 
    \item Recent focus has been on computing statistics predicted by theory from the data and adding as figures into the manuscript 
    \item Draft is at v0.5 stage with theoretical results in place and sections reserved for data analysis 
  \end{itemize}

 \noindent \textbf{AIM 3:} A population genetic history of the Kodava population\\
  \begin{itemize}
    \item Data freeze generated from $N = 90$ Kodava individuals from North-America (WGS data (Average coverage : $\sim 5\text{x}$))
    \item Major questions related to ancestry:
      \begin{enumerate}
        \item Do the Kodava have a higher proportion of Ancestral North Indian (ANI) ancestry relative to nearby South Indian populations?
        \item Are the best proxies for regional admixture consistent with Kodava oral history?
        \item Is there potential signal of actionable disease variation relevant to other South Indian populations? 
 \end{enumerate}
    \item From PCA / STRUCTURE there is little evidence that the Kodava harbor substantially higher ANI ancestry (supported by more formal $f_3/f_4$-tests as well)
    \item Analyses planned to explore identity-by-descent within the community and runs-of-homozygosity as well. 
 \end{itemize}

\bibliography{report2020}
\bibliographystyle{plain}

\vspace*{2em}
\begin{figure}[H]
  \centering
  \includegraphics{phd-timeline.pdf}
  \caption*{Proposed time-allocation for next months}
\end{figure}


\end{document}
